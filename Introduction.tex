\chapter{Introduction}

In this work, I implement a sketching interface that allows the user to create drawings in three dimensions.
The interface is capable of using modern input devices to approximate the act of real world sketching as closely as possible.
These devices include a display capable of multi-touch input, as well as a special electronic pen which relays extra information
not sent through regular touch input.
For the sketch itself, I implement a spline-based data structure in order to store high quality, three dimensional strokes, that can be zoomed in without loss in quality.
I also implement a three-dimensional line rendering library, as native line rendering implementations are poor quality.

When creating a three-dimensional model, such as a building or a character in an animated film, much of the work is done on a computer using Computer-Aided Design (CAD) software.
However, the initial designs are still done using two-dimensional sketches, rough drawings not intended as the finished work.
Sketches are generally not highly detailed works, as they intend to only capture the essentials of a final design.
Through a number of rough sketches, a three-dimensional form can be created through the representations of perspective and volume.
This two-dimensional information is then used as a reference when designing the final three-dimensional model.
Details of modern approaches to content creation on a computer are given in Chapter 2.

In architecture, there is a push towards creating buildings that reduce energy consumption though heating, cooling, and lighting.
If an architect has a three dimensional model, it can be analyzed to predict how well it uses energy.
However, if the building has poor results, the architect can only make superficial changes to the structure of the building, since much of the design process has already been approved.
If we are able to digitize the early phase design process such that a rudimentary three-dimensional model can be made, architects can analyze their models earlier in the design process, and use this information to better design energy efficient buildings.

Sketching is an old method of expressing ideas, and has a variety of techniques associated with the practice.
Many professionals have been reluctant to use computer software, because the skills they have used and trained themselves in do not transfer to the digital medium.
In recent years, technology has advanced to where the creation of specialized user input devices can allow better emulation of traditional sketching techniques.
These devices are explored in Chapter~\ref{ch:input}.

In our interface, a user sketches through an interaction window displaying a scene.
This scene contains geometry of some form that the user can draw on.
Every time the user draws on an object, a stroke is created.
In Chapters 4 and 5, we discuss how the rough, two dimensional user input is transformed into a high-quality, three-dimensional spline curve.
This spline is rendered in real time using OpenGL.
However, the native support for displaying curves is limited, as many artifacts appear in the final curve when naively attempting to render them.
Details on the approach we use to display our curves in high quality is discussed in Chapter 6.