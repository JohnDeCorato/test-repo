\documentclass[12pt]{report}
\usepackage[utf8]{inputenc}
\usepackage{graphicx}
\usepackage{subcaption}
\usepackage{setspace}
\usepackage[english]{babel}
\usepackage{wrapfig}
\usepackage[font=small,labelfont=bf]{caption}

\doublespacing

\graphicspath{{Images/}}
\title{Thesis}
\author{John DeCorato}
\date{ }

\setlength{\parindent}{3em}
\setlength{\parskip}{1em}

\begin{document}

\chapter{Splines Chapter}

\section{Splines}
A spline is a mathematical function that is piecewise defined by polynomial functions. 
Spline functions also have high degrees of smoothness where the pieces connect, meaning that the spline overall is continuously differentiable after multiple derivatives.
Splines are most commonly used for interpolation between data points.
This is because the interpretation error can be made small when even low order polynomial functions are used for the spline.
The most common type of spline used, as well as the one used for this project, is a B-spline.

\subsection{B-Splines}


\end{document}